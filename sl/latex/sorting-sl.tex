\chapter{Algoritmi za urejanje}
\chaplabel{sorting}

\translatedby{Rok Samsa}{sl}

Hitro urejanje ali\textit{quicksort} algoritem je še en klasični ``deli in vladaj'' algoritem. V nasprotju z algoritmom zlivanja (mergesort), kateri združuje po rešitvi dveh podproblemov, algoritem hitrega urejanja počne vse svoje delo vnaprej.

\vspace{1em}

Algoritem lahko preprosto opišemo tako: Izberemo naključni delilni element, ki ga imenujemo pivot, x. Dobimo ga iz a; Particija a je sestavljena iz sklopa elementov manjših kot x, sklopa elementov enakih kot x in niz elementov večjih od x; na koncu pa rekurzivno razvrstimo prvi in tretji sklop števil v tej particiji. Primer je prikazan na sliki 11.3.

\vspace{1em}

\centering Slika 11.3: Primer izvedbe algoritma hitrega urejanja (a, 0, 14, c)

\raggedright 

\vspace{1em}

Vse to je narejeno v enem koraku, tako da namesto ustvarjanja kopij urejenih podseznamov, quickSort(a,  i,  n,  c) metoda razvršča samo podseznam a[i],..., a[i + n - 1]. Prvotno kli čemo to metodo kot quickSort(a, 0, a.length, c). 

\vspace{1em}

V središču quicksort algoritma je algoritem delitve na mestu. Ta algoritem, brez uporabe dodatnega prostora, zamenja elemente v a in izračuna indekse p in q tako da:

$$
a[i] \left\lbrace
\begin{array}{ll}
%< x \hspace{1em} \text{č}e \enspace je \enspace 0 \leq 1 \leq p\\
%= x \hspace{1em} \text{č}e \enspace je \enspace p < 1 < q\\
%> x \hspace{1em} \text{č}e \enspace je \enspace q \leq 1 \leq n - 1\\
\end{array}
\right.
$$

\vspace{1em}

Ta delitev, ki se opravi z ``while'' zanko v sami kodi, deluje s ponavljajočim povečanjem p-ja in zmanjševanjem q-ja ob ohranjanju prvega in zadnjega od teh pogojev (p in q). V vsakem koraku element na položaju j je premaknjen levo na prvo mesto, ali pa je premaknjen na zadnje mesto. V prvih dveh primerih, je j povečan, v zadnjem primeru j ni povečan, zato ker nov element na položaju j še ni bil obdelan.

\vspace{1em}

Quicksort algoritem je zelo tesno povezan z naključnim binarnim iskalnim drevesom, opisan v poglavju 7.1. Dejansko, če poženemo quicksort algoritem nad n različnimi elementi, potem je quicksort rekurzivno drevo naključno iskalno drevo. Da bi to videli, se moramo spomniti, kako gradimo naključno binarno iskalno drevo. Najprej naključno izberemo element x in ga postavimo za koren drevesa. Takoj za tem, vsak naslednji element primerjamo z x-om. Manjše elemente postavljamo v levo stran poddrevesa in večje elemente v desno stran poddrevesa.

\vspace{1em}

S tem algoritmom, izberemo nakjučni element x in takoj za tem primerjamo vse s tem x-om. Najmanjše elemente postavimo na začetek polja in večje elemente postavimo na konec polja. Quicksort algoritem nato rekurzivno uredi začetek in konec polja, medtem ko naključno iskalno drevo rekurzivno vstavi manjše elemente v levo poddrevo korena in večje elemente v desno poddrevo korena.

\vspace{1em}

Zgornje ujemanje med naključnim binarnim iskalnim drevesom in algoritmom hitrega urejanja, lahko uporabimo za Lemo 7.1 

\vspace{1em}

\raggedright 
\textbf{Lemma 11.1.} \emph{Ko kličemo algoritem quicksort za urejanje polja, ki vsebuje cela števila} $0,..., n-1$ \emph{pričakovano število primerjav elementa s pivot elementom je} $H_{i+1}+H^{n-i}$.

\vspace{1em}

Malo seštevanja harmoničnih števil nam daje naslednji izrek o času delovanja, katerega porabi algoritem:

\vspace{1em}

\raggedright 
\textbf{Izrek 11.2.} \emph{Ko quicksort algoritem uporabimo za urejanje polja z n različnimi elementi, pričakujemo največje število opravljenih primerjav} 2nlnn +O(n).

\vspace{1em}

\emph{Proof.} Naj bo T število primerjav opravljenih z algoritmom quicksort, ko razvrš ča n razlčne elemente. Z uporabo Leme 11,1, imamo:

\vspace{1em}

Izrek 11.3 opisuje primer, kjer so razvrš čeni elementi vsi različni. Ko vhodna matrika, a, vsebuje podvojene elemente, pričakovani čas delovanja za hitro urejanje ni nič slabši, in je lahko celo boljši; vedno ko je podvojeni element x izbran kot element pivot a, vse pojavitve x-a se združijo in jih kasneje ne vključimo v enem od dveh podproblemov.

\vspace{1em}

\raggedright 
\textbf{Izrek 11.3.} \emph{Quicksort(a, c) metoda ima pričakovani čas izvedbe O ($n log n$) in pričakovano število primerjav, ki jih opravi, je v večini} $2n\ln n$ + \emph{O (}n\emph{).}

\section{Diskusija in Naloge}

Sortiranje je osnovni algoritemski problem v računalništvu in ima dolgo zgodovino.
Knuth \cite{k97v3} pripisuje alogritem sortiranja z zlivanjem
von Neumann(1945). Hitro urejanje je last Hoare \cite{h61}.
Originalno urejanje s kopico je last Williams \cite{w64}, ampak verzija, ki je tu 
predstavljena(v kateri se kopica gradi iz spodaj nazvgor v $O(#n#)$ času) je last Floyd \cite{f64}.
Spodnje meje za sortiranje s primerjavami se zdijo folklorne. Naslednja tabela
povzame izvedbo teh algoritmov za urenjanje s primerjavami:

\begin{center}
  \begin{tabular}{|l|r@{}l@{ }l|l|} \hline
    & \multicolumn{3}{c|}{primerjave} & na mestu  \\ \hline
    Urejanje z zlivanjem & $#n#\log #n#$ & &  najslabši primer & Ne  \\
    Hitrjo urejanje & $1.38#n#\log #n#$ & ${}+ O(#n#)$ & pričakovano & Da \\
    Urejanje s kopico & $2#n#\log #n#$ & ${}+ O(#n#)$ & najslabši primer & Da \\ \hline
  \end{tabular}
\end{center}

Vsi te alogiritmi urejanje s primerjanjem imajo svoje prednosti in slabosti.
Urejanje z zlivanjem naredi najmanj primerjav in se ne zanaša na naključnost.
Na žalost, uporablja pomožno tabelo med fazo zlivanja. Dodeljevanje pomnilnika 
tej tabeli je lahko drago in ima potencial, da je to usodnno za algoritem, če je
količina spomina omejena. Hitro urejanje je algoritem urejanja \emph{na mestu}
\index{in-place algorithm}%
in je blizu na drugem mestu v številu primerjav, ampak je naključno, zato čas 
izvajanja ni vedno zagotovljen. Urejanje s kopico naredi največ primerjav, ampak je 
urejanje na mestu in je deterministično.

Obstaja en primer, v katerem je urejanje s kopico očiten zmagovalec; to se zgodi pri sortiranju
povezanega seznama. V tem primeru, ne potrebujemo pomožne tabele; dva urejena povezana seznama, 
se zelo lahko zljieta v en urejen povezan seznam z uporabo manipulacije kazalcev (glej
\excref{list-merge-sort}).

Algoritma urejanja s štetjem in urejanja po delih opisana tu, je last 
Seward \cite[Section~2.4.6]{s54}. Ampak različice urejanja po delih so
v uporabi že od 20 let 20. stoletja, za urejanje luknjanih kartic
z uporabo strojev za urejanje luknjanih kartic. Te stroji lahko uredijo 
kup kartic v dva kupa, glede na obstoj(ali neobstoj) ljuknice na specifični
lokaciji na kartici. Ponovitev tega procesa, za drugo
luknjico nam da implementacijo urejanja po delih.

Na koncu, opazimo, da urejanje s štetjem in po delih lahko uporabimo, za 
urejanje drugih vrst številk razen ne negativnih celih števil. 
Enostavne spremembe urejanja s štetjem lahko sortirajo cela števila
v kateremkoli intervalu $\{a,\ldots,b\}$, v $O(#n#+b-a)$ času.  Podobno, urejanje 
po delih lahko ureja cela števila na enakem intervalu v $O(#n#(\log_{#n#}(b-a))$ času.
Na koncu, lahko oba algoritma uporabimo za urejanje števil s plavajočo vejico v 
IEEE 754 formatu plavjoče vejice. To lahko naredimo zato, ker je IEEE format narejen
tako, da dovoljuje primerjavo dveh števil s plavajočo vejico glede na njuni vrednosti, 
kot če bi bili celi števili v predznačeni dvojiški predstavitvi \cite{ieee754}.

\begin{exc}
  Ilustriraje izvedbo urejanje z zlivanjem in urejanja s kopico na vhodni tabeli,
  ki vsebuje $1,7,4,6,2,8,3,5$. Naredite vzorčno ilustracijo ene možnosti izvede
  hitrega urejanja na isti tabeli.
\end{exc}

\begin{exc}\exclabel{list-merge-sort}
  Implementirajte verzijo algoritma za urejanje z zlivanjem, ki sortirajo dvojno povezan seznam, brez uporabe
  pomožne tabele. (Glej \excref{dllist-sort}.) 
\end{exc}

\begin{exc}
  Nekatere implementacije #quickSort(a,i,n,c)# vedno uporabljajo #a[i]#
  kot pivot.  V primeru, da je vhodna tabele dolžine #n# v kateri taka implementacija izvede
  $\binom{#n#}{2}$ primerjav.
\end{exc}

\begin{exc}
  Nekatere implementacije #quickSort(a,i,n,c)# vedno uporabljajo #a[i+n/2]#
  kot pivot. V primeru, da je vhodna tabele dolžine #n# v kateri taka 
  implementacija izvede $\binom{#n#}{2}$ primerjav.
\end{exc}

\begin{exc}
  Pokažite, da za katerokoli implementacijo #quickSort(a,i,n,c)#,
  ki izbere pivot deterministično, brez da pogleda katerokoli vrednost 
  v $#a[i]#,\ldots,#a[i+n-1]#$, obstaja vhodna tabela dolžine #n#,
  katera povzroči to implementacijo, da naredi $\binom{#n#}{2}$ primerjav.
\end{exc}

\begin{exc}
  Načrtujte #Comparator#, #c#, katerega lahko podate kot argument funkciji
  #quickSort(a,i,n,c)#, kateri bi povzročil $\binom{#n#}{2}$ primerjav.  
  (Namig: Vašemu Comparator-ju ni potrebno gledati vrednosti, ki se primerjajo.)
\end{exc}

\begin{exc}
  Analizirajte pričakovano število primerjav, ki jih naredi Quicksort, malo bolj pazljivo, kot
  dokaz \thmref{quicksort}. Dokažite, da je pričakovano število primerjav $2#n#H_#n# -#n# + H_#n#$.
\end{exc}

\begin{exc}
  Opišite vhodno tabelo, ki povzroči, da urejanje s kopico, naredi največ
  $2#n#\log #n#-O(#n#)$ primerjav. Utemeljite vaš odgovor.
\end{exc}

\javaonly{
\begin{exc}
  Implementacija sortiranja s kopico, ki je opisana tukaj, uredi elemente v obratni 
  vrstni red in nato  tabelo. Ta zadnji korak, lahko izputimo, če definiramo nov #Comparator#,
  ki negira rezultat vhoda #Comparatorja# #c#. Razložite zakaj to nebi
  bila dobra optimizacija. (Namig: Pomislite koliko negacij bi bilo potrebno v razmerju s koliko časa potrebujemo, da obrnemo tabelo.)
\end{exc}
}

\begin{exc}
  Najdite nek drug par premutacij $1,2,3$ , ki nisto pravilno urejene
  z drevesom primerjav v \figref{comparison-tree-2}.
\end{exc}

\begin{exc}\exclabel{log-factorial}
  Dokažite, da $\log #n#! = #n#\log #n#-O(#n#)$.
\end{exc}

\begin{exc}
  Dokažite, da dvojiško drevo s $k$ listi ima višino najmanj $\log k$.
\end{exc}

\begin{exc}\exclabel{randomized-lower-bound}
  okažite, da če izberemo naključen list iz dvojiškega drevesa s $k$ listi,
  potem je pričakovana višina lista, najmanj $\log k$.
\end{exc}

\begin{exc}
  Implementacija #radixSort(a,k)# podana tukaj, deluje ko vhodna 
  tabela, #a# vsebuje samo \javaonly{ne negativna} cela števila.
  \javaonly{Razširite to implementacijo, tako da deluje tudi, ko #a# vsebuje negativna in ne negativna cela števila.}
  \cpponly{Napišite verzijo, ki deluje za predznačena cela števila.}
\end{exc}