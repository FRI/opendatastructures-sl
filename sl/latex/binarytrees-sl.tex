\chapter{Dvojiška drevesa}
\chaplabel{dvojiškadrevesa}
\translatedby{Erik Šimer}{sl}

To poglavje predstavlja eno najbolj temeljnih struktur v računalništvu
: dvojiška drevesa.Uporaba besede \emph{drevo}
\index{tree}%
\index{tree!binary}%
\index{binary tree}%
prihaja iz dejstva, ko jih rišemo, da je risba podobna drevesom iz gozda.
Obstaja veliko načinov definiranja binarnega drevesa.
 Matematično je  \ emph {binarno drevo} povezan,
neusmerjen, končni graf brez ciklov.

Za večino aplikacij v računalništvu, so binarna drevesa \ emph {zakoreninjena:}
\index{tree!rooted}%
\index{rooted tree}%
Posebno vozlišče, # r#, v največ drugi stopnji, se imenuje \ emph {koren}
drevesa. Za vsako vozlišče, $#u#\neq #r#$, se  drugo vozlišče na
poti od # u # do # R # imenuje \ emph {starš} od #u# .
\ index {parent}%
Vsa druga vozlišča, ki mejijo na # u # imenujemo \ emph {otrok}
\index{child} od #u#.Večina
dvojiških dreves ki nas zanimajo, so \ emph {urejena},
\index{ordered tree}%
\index{tree!ordered}%
Tako ločimo
med \ emph {levim otrokom} in  \ emph {desnim otrokom} od #u#.
\index{left child}%
\index{child!left}%
\index{right child}%
\index{child!right}%

V ilustraciji, so dvojiška drevesa običajno sestavljena iz korena
navzdo. Koren je  na vrhu risbe, ki ima levega in desnega otroka.Levi otrok ej na levi strani, desni pa na desni strani
(\ figref {bintree-orientation}). Na primer \ figref {binary-tree}. Kaže
binarno drevo z devetimi vozlišči.

\begin{figure}
  \begin{center}
    \includegraphics[scale=0.90909]{figs/bintree-traverse-1} 
  \end{center}
  \caption[Parent, left child, and right child]{Starš, levi otrok, desni otrok vozlišča  #u#
   v #BinaryTree#.}
  \figlabel{bintree-orientation}
\end{figure}


\begin{figure}
  \begin{center}
    \begin{tabular}{cc}
      \includegraphics[width=\HalfScaleIfNeeded]{figs/bintree-1} &
      \includegraphics[width=\HalfScaleIfNeeded]{figs/bintree-2} \\
      (a) & (b)
    \end{tabular}
  \end{center}
  \caption{Binarno drevo (a)~devet vozlišč in (b)~deset zunanjih vozlišč.}
  \figlabel{binary-tree}
\end{figure}

Ker so dvojiška drevesa tako pomembna, so za njih razvili določeno terminologijoi:
 \ emph {globina}
\index{depth}%
vozlišča, #u#, je v binarnem drevesu
dolžina poti od #u# do korena drevesa. Če vozlišče, #w#,
je na poti od #u# do # r#, potem #w# imenujemo \ emph {prednik}
\index{ancestor}%
od #u# in #u# pa \emph{potomec}
\index{descendant}%
od #w# .\ Emph {poddrevo} od
vozlišče, #u#, je binarno drevo, ki ima korenine v #u#  in vsebuje vse
od potomce od # u# .\ Emph {višina}
\index{height!in a tree} vozlišča, #u#, je dolžina
najdaljše poti od # u#  do enega od njenih potomcev.\ Emph {višina} od
\index{height!of a tree}%
drevesa je višina njegovega korena.
Vozlišče, #u#, je \ emph {list}
\index{leaf}%
če nima nobenega otroka.

Včasih mislimo, da so drevesa utrjena z \ emph {zunanjimi
vozlišči}. Vsako vozlišče, ki nima levega otroka ima zunanje
vozlišče kot svojega levega otroka, in ustrezno vsakemo vozlišče, ki
nima pravega otroka ima zunanjo vozlišče kot svojega pravega otroka (glejte
\figref{binary-tree}.b). Z indukcijo lahko enostavno preverimo, da binarno drevo z 
$#n#\ge 1$ pravimi vozlišči ima  $#n#+1$  zunanjih vozlišč.


\section{#Binarno Drevo#: Enostavno binarno drevo}

\index{BinaryTree@#BinaryTree#}%
Najenostavnejši način, predstavitve vozlišča, #u#, v binarnem drevesu je
izrecno shranjevanje (največ treh) sosedeov od #u#:
\javaimport{ods/BinaryTree.BTNode<Node}
\cppimport{ods/BinaryTree.BTNode}
Ko eden odtreh sosedov ni prisoten, ga nastavimo na #nil#.
Na ta način sta oba zunanja vozlišča drevesa in starš korena
vrednosti #nil#.

Binarno drevo se lahko zastopa kot
\javaonly{referenca}\cpponly{pointer} do svojega vozlišča korena, #r#:
\codeimport{ods/BinaryTree.r}

Globino vozlišča, #u#, lahko izračunamo tako, da štejemo korake od #u# do korena:
\codeimport{ods/BinaryTree.depth(u)}


\subsection{Rekurzivni algoritmi}

\index{recursive algorithm}%
Z uporabo rekurzivnih algoritmov je zelo enostavno izračunati dejstva o
binarnih drevesih. Na primer, za izračun velikosti (število vozlišč)
binarnega drevesa, ki je  zakoreninjen v vozlišču #u#, tako da rekurzivno izračunamo velikost
dveh poddreve, ki so  zakoreninjena na otroke od #u#, nato povzamemo te velikosti, in dodamo eno:

\codeimport{ods/BinaryTree.size(u)}

Za izračun višine vozlišča #u# moremo izračunati višino #u#- jevih dveh poddreves,
vzemiti največjega  in mu dodamo:

\codeimport{ods/BinaryTree.height(u)}

\subsection{Obiskovanje Binarnega drevesa}
\seclabel{bintree:traversal}

\index{traversal!of a binary tree}%
\index{tree traversal}%
\index{binary-tree traversal}%
Prejšnja dva algoritma iz prejšnjega odseka uporabljata rekurzijo, za obisk
vseh vozlišč v binarnem drevesu. Vsak od njih obišče vozlišča
binarnega drevesa v istem vrstnem redu kot naslednja koda:
\codeimport{ods/BinaryTree.traverse(u)}

Z uporabo rekurzije, lahko na ta način proizvajamo zelo kratko in preprosto kodo, lahko pa je taka koda zelo problematična.
Največja globina rekurzijska je podana z
največjo globino vozlišča v dvojiškem drevesu, tj, višina drevesa.
Če je višina drevesa zelo velika, potem lahko taka rekurzija 
porabi veliko več pomnilnika na skladu, kot ga je na voljo.

Za obhod binarnega drevesa brez rekurzije, lahko uporabite algoritem, ki se 
zanaša na to, da ve iz  kje je prišel, in kam bo šel. Glej
\figref{bintree-traverse}.  Če pridemo do vozlišča #u# od #u.starš#,
potem obiščemo #u.left#. Če pridemo do #u#
od #u.left#, potem obiščemo #u.right#. Če prispemo
 na #u# iz #u.right# , potem smo končali z obiskovanjem #u# -jevih poddreves,
in se tako vrnemo na #u.parent#. Naslednja koda izvaja to
idejo, ki vključuje ravnanje v primerih, ko katera koli od #u.left#,
#u.right# ali #u.parent# je #ni#l:
\codeimport{ods/BinaryTree.traverse2()}

\begin{figure}
  \begin{center}
    \begin{tabular}{cc}
      \includegraphics[scale=0.90909]{figs/bintree-traverse-2}
      \includegraphics[scale=0.90909]{figs/bintree-3}
    \end{tabular}
  \end{center}
  \caption[Traversing a BinaryTree]{Trije primeri, ki se pojavijo na vozlišču
    #u# kadar obhodimo binarna drevesa,ki niso rekurzivna}
  \figlabel{bintree-traverse}
\end{figure}

Enake primere, ki jih lahko izračunamo z rekurzivnimi algoritmi, lahko izračunamo z iterativnimi algoritmi. 
Na primer, za izračun velikosti drevesa hranimo števec #n#, in nižamo #n# vsakič ko obiščemo novo vozlišče.
\codeimport{ods/BinaryTree.size2()}

V nekaterih implementacijah binarnih dreves, se #starš# ne uporablja.
V takih primerih, lahko še vedno uporabimo ne rekurzivno izvedbo, vendar
mora taka izvedba uporabljati  # seznam # (ali #sklad#), saj bi tako lahko
spremljali pot od trenutnega vozlišča do korena.

Posebna vrsta prečkanja, ki ne ustreza vzorcu zgoraj navedene funkcije je
 \emph{prvi-v-širino}.
\index{breadth-first traversal}%
\index{traversal!breadth-first}%
V prvi-v-širino obhodu, so vozlišča obiskana stopnja-postopnjo,
 pri kateremu začnemo v korenu in nadaljujemo navzdol,
 kjer obiskujemo vsako vozlišče od levega proti desni (glej
\figref{bintree-bfs}). 
To je podobno načinu branja strani v Angleškem jeziku.
Prvi-v-širino obhod je implementiran z uporabo vrste #q#, ki na začetku vsebuje samo koren  #r#.
Na vsakem koraku, vzamemo naslednje vozlišče #u# iz #q# , nato procesiramo #u#, in dodamo #u.left#
in #u.right# (če niso #nil#) v #q#:
\codeimport{ods/BinaryTree.bfTraverse()}

\begin{figure}
  \begin{center}
    \includegraphics[scale=0.90909]{figs/bintree-4}
  \end{center}
  \caption{Med prvi-v-širino obhodu,  so vozlišča v binarnem drevesu obiskana po principu stopnja-po-stopnjo in levo-proti-desni za vsako stopnjo.}
  \figlabel{bintree-bfs}
\end{figure}



