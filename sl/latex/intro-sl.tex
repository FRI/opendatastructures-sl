\translatedby{Timotej Kos}{sl}


\chapter{Uvod}
\pagenumbering{arabic}

Vsak računalniški predmet na svetu vključuje snov o podatkovnih strukturah in algoritmih. Podatkovne strukture so \emph{tako} pomembne; izboljšajo kvaliteto našega življenja in celo vsakodnevno rešujejo življenja. Veliko multimiljonskih in nekaj multimiljardnih družb je bilo ustanovljenih na osnovi podatkovnih struktur.

Kako je to možno? Če dobro pomislimo ugotovimo, da se s podatkovnimi strukturami srečujemo povsod. 

\begin{itemize}

\item Odpiranje datoteke:Podatkovne strukture
datotečnega sistema \index{datotečni sistem} se uporabljajo za iskanje delov datoteke na disku. To ni preprosto; diski vsebujejo stotine miljonov blokov, vsebina datoteke pa je lahko spravljena v kateremkoli od njih.
 
\item Imenik na telefonu: Podatkovna struktura se uporabi za iskanje telefonske številke v imeniku \index{imenik}, glede na delno informacijo še preden končamo z vnosom iskalnega pojma. To ni preprosto: naš imenik lahko vsebuje ogromno informacij-vsi, ki smo jih kadarkoli kontaktirali prek telefona ali elektronske pošte-telefon pa nima zelo hitrega procesorja ali veliko spomina.  

\item Vpis v socialno omrežje: \index{socialna omrežja} Omrežni strežniki uporabljajo naše vpisne podatke za vpogled v naš račun. To ni preprosto: največja socialna omrežja imajo stotine miljonov aktivnih uporabnikov.

\item Spletno iskanje: \index{spletno iskanje} Iskalniki uporabljajo podatkovne strukture za iskanje spletnih strani, ki vsebujejo naše iskalne pojme. To ni preprosto: v Internetu je čez 8.5 miljard spletnih strani, kjer vsaka vsebuje veliko potencialnih iskalnih pojmov.

\item Številke za klice v sili(1-1-2, 1-1-3): \index{številke za klic v sili} Omrežje za storitve klicev v sili poišče našo telefonsko številko v podatkovni strukturi, da lahko gasilna, reševalna in policijska vozila pošlje na kraj nesreče brez zamud. TO je pomembno; oseba, ki kliče mogoče ni zmožna zagotoviti pravilnega naslova in zamuda lahko pomeni razliko med življenjem in smrtjo. 

\end{itemize}


\section{Zahteva po učinkovitosti}

V tem poglavju bomo pogledali operacije najbolj pogosto uporabljanih podatkovnih struktur. Vsak s vsaj malo programerskega znanja bo videl, da so te operacije lahke za implementacijo. Podatke lahko shranimo v polje ali povezan seznam, vsaka operacija pa je lahko implementirana s sprehodom čez polje ali povezan seznam in morebitnim dodajanjem ali brisanjem elementa.

Takšna implementacija je preprosta vendar ni učinkovita. Ali je to sploh pomembno? Računalniki postajajo vse hitrejši zato je mogoče takšna implementacija dovolj dobra. Za odgovor naredimo nekaj izračunov. 

\paragraph{Število operacij:}
Predstavljajte si program z zmerno velikim naborom podatkov, recimo enim milijonom ($10^6$) elementov. V večini programov je logično sklepati, da bo program pregledal vsak element vsaj enkrat. To pomeni, da lahko pričakujemo vsaj milijon ($10^6$) iskanj. Če vsako od teh $10^6$ iskanj pregleda vsakega od $10^6$ elementov je to skupaj $10^6\times 10^6=10^{12}$ (tisoč milijard) iskanj. 

\paragraph{Procesorske hitrosti:} 
V času pisanja celo zelo hiter namizni računalnik ne more opraviti več kot milijardo ($10^9$) operacij na sekundo. \footnote{Računalniške hitrosti se merijo v nekaj gigaherzih (milijarda ciklov na sekundo), kjer vsaka operacija zahteva nekaj ciklov.} To pomeni, da bo ta program porabil najmanj $10^{12}/10^9 = 1000$ sekund ali na grobo 16 minut in 40 sekund. Šestnajst minut je v računalniškem času ogromno, človeku pa bo to pomenilo veliko manj (sploh če si vzame odmor). 

\paragraph{Večji nabori podatkov:} Predstavljajte si podjetje kot je Google, \index{Google}, ki upravlja z čez 8.5 miljard spletnimi stranmi. Po naših izračunih bi kakršnakoli poizvedba med temi podatki trajala najmanj 8.5 sekund. Vendar vemo, da ni tako; spletna iskanja se izvedejo veliko hitreje kot v 8.5 sekundah, hkrati pa opravljajo veliko zahtevnejše poizvedbe, kot samo iskanje ali je določena stran na seznamu ali ne.  V času našega pisanja, Google prejme najmanj $4,500$ poizvedb na sekund kar pomeni, da bi zahtevalo najmanj $4,500\times 8.5=38,250$ zelo hitrih stežnikov samo za vzdrževanje.  

\paragraph{Rešitev:}
Ti primeri nam povejo, da preproste implementacije podatkovnih struktur ne delujejo ko sta tako število elementov, #n#, v podatkovni strukturi kot tudi število operacij, $m$, opravljenih na podatkivni strukturi, velika. V takih primerih je čas (merjen v korakih) na grobo $#n#\times m$.

Rešitev je premišljena organizacija podatkov v podatkovni strukturi tako, da vsaka operacija ne zahteva poizvedbe po vsakem elementu. Čeprav se sliši nemogoče bomo spoznali podatkovne strukture, kjer iskanje zahteva primerjavo samo dveh elementov v povpreču, neodvisno od števila elementov v podatkovni strukturi. V naštem računalniku, ki opravi milijardo operacij na sekundo zahteva iskanje v podatkovni strukturi, ki vsebuje milijardo elementov (ali trilijardo, štirijardo ali celo petiljardo elementov), samo $0.000000002$ sekund. 

Pogledali bomo tudi implementacije podatkovnih struktur, ki hranijo elemente v vrstnem redu, kjer število poizvedenih elementov med operacijo raste zelo počasi v odvisnosti od števila elementov v podatkovni strukturi. Naprimer, lahko vzdržujemo sortiran niz milijarde elementov, med poizvedbo do največ 60 elementov med katerokoli operacijo. V naštem računalniku, ki opravi milijardo operacij na sekundo zahteva iskanje v podatkovni strukturi, ki vsebuje milijardo elementov, samo $0.000000002$ sekund. 

Preostanek tega poglavja vsebuje kratek pregled glavnih pojmov, ki se uporabljajo skozi celotno knjigo. \secref{vmesniki} opisuje vmesnike, ki so implementirani z vsemi podatkovnimi strukturami opisanimi v tej knjigi in je smatran kot obvezno branje. Ostala poglavja govorijo o:
\begin{itemize}
\item pregled matematičnega dela, ki vključuje eksponente, logaritme, fakultete, asimptotično (veliki O) notacijo, verjetnost in naključnost;
\item računski model;
\item pravilnost, časovna zahtevnost in prostorska zahtevnost;
\item pregled ostalih poglavij; in
\item vzorčne kode in navodila za pisanje
\end{itemize}

Bralec z ali brez podlage na tem področju jih lahko enostavno preskoči za zdaj in se vrne kasneje če bo potrebno. 



\translatedby{Bostjan Vizintin}{sl} 
\section{Matematično ozdaje}


V tem poglavju, so opisane nekatere matematične notacije in orodja, ki so uporabljena v knjigi, vključno z logaritmi, veliko-O notacijo in verjetnostno teorijo. Opis ne bo natančen in ni mišljen kot uvod. Vsi bralci, ki mislijo da jim manjka osnovno znanje, si več lahko preberejo in naredijo nekaj nalog iz vstreznih poglavji iz zelo dobre in zastonj knjige o znanosti iz matematike in računalništva 
\cite{llm11}.

\subsection{Eksponenti in Logaritmi}

\index{eksponenti}%

Izraz $b^x$ označuje število $b$ na potenco $x$. Če je $x$ pozitivno celo število, potem je to samo število $b$ pomnoženo samo s seboj $x-1$ krat:

\[
b^x = \underbrace{b\times b\times \cdots \times b}_{x} \enspace .
\]

Ko je $x$ negativno celo število, je $b^x=1/b^{-x}$. Ko je $x=0$, $b^x=1$.
Ko $b$ ni celo število, še vedno lahko definiramo potenciranje v smislu eksponentne funkcije $e^x$ (glej spodaj), ki je definirana v smislu eksponentne serije, vendar jo je najboljše prepustiti računskemu besedilu.

\index{logaritm}%

V tej knjigi se izraz $\log_b k$ označuje \emph{logaritm z osnovo-$b$} od $k$. To je edinstvena vrednost $x$ za katero velja
\[
b^{x} = k \enspace .
\]

Večina logaritmov v tej knjigi ima osnovo 2 (\emph{binarni logaritmi}).

\index{binary logarithm}%
\index{logarithm!binary}%

Za te logaritme, izpustimo osnovo, tako je $\log k$ skrajšan izraz za $\log_2 k$.

Neformalen ampak uporaben način je, da mislimo na $\log_b k$, kot število, koliko krat moramo deliti $k$ z $b$, preden bo rezultat manjši ali enak 1. Na primer, ko izvedemo binarno iskanje vsaka primerjava zmanjša število možnih odgovorov za faktor 2. To se ponavlja, dokler nam ne preostane samo en možen odgovor. Zato je število primerjav pri binarnem iskanju nad največ $n+1$ podatki enako največ $\lceil\log_2(n+1)\rceil$.





\index{natural logarithm}%
\index{logarithm!natural}%

V knjigi se večkrat pojavi tudi \emph{naravni logaritm}. Pri naravnem logaritmu uporabimo notacijo $\ln k$ ki označuje $\log_e k$, kjer je $e$ --- \emph{Eulerjeva konstanta} --- podan na naslednji način: 

\index{Euler's constant}%
\index{e@$e$ (Euler's constant)}%
\[
e = \lim_{n\rightarrow\infty} \left(1+\frac{1}{n}\right)^n
\approx 2.71828 \enspace .
\]

Naravni logaritem pride v poštev pogosto, ker je vrednost zelo pogostega integrala:

\[
\int_{1}^{k} 1/x\,\mathrm{d}x = \ln k \enspace .
\]

Dve najbolj pogoste operacije ki jih naredimo nad logaritmi sta, da jih umaknemo iz eksponenta:

\[
b^{\log_b k} = k
\]
in zamenjamo osnovo logaritma:
\[
\log_b k = \frac{\log_a k}{\log_a b} \enspace .
\]

Na primer, te dve operaciji lahko uporabimo za primerjavo naravnih in binarnin logaritmov.
\[
\ln k = \frac{\log k}{\log e} = \frac{\log k}{(\ln e)/(\ln 2)} = 
(\ln 2)(\log k) \approx 0.693147\log k \enspace .
\]

\subsection{Fakulteta}
\seclabel{fakulteta}

\index{factorial}%

V enem ali dveh delih knjige je uporabljena \emph{fakulteta}
Za nenegativna cela števila $n$, je uporabljena notacija $n!$ (izgovorjena kot ``$n$ fakulteta'') in pomeni naslednje:
\[
n! = 1\cdot2\cdot3\cdot\cdots\cdot n \enspace .
\]
Fakulteta se pojavi, ker je $n!$ število različnih permutaciji, naprimer zaporedja $n$ različnih elementov.


\index{permutacija}%

Za poseben primer $n=0$, je $0!$ definiran kot 1.

\index{Stirlingov približek}%
Vrednost  $n!$ je lahko približno določena z uporabo \emph{Stirlingovega približka}: 

\[
n! 
= \sqrt{2\pi n}\left(\frac{n}{e}\right)^{n}e^{\alpha(n)} \enspace ,
\]
kjer je
\[ 
\frac{1}{12n+1} < \alpha(n) < \frac{1}{12n} \enspace .
\]
Stirlingov približek prav tako približno določa $\ln(n!)$:
\[
\ln(n!) = n\ln n - n + \frac{1}{2}\ln(2\pi n) + \alpha(n)
\]
(V bistvu je Stirlingov približek najlažje dokazan z približevanjem
$\ln(n!)=\ln 1 + \ln 2 + \cdots + \ln n$ z integralom
$\int_1^n \ln n\,\mathrm{d}n = n\ln n - n +1$.)

\index{binomski koeficient}%
V relaciji s fakultetami so \emph{binomski koeficienti}.
Za nenegativna cela števila $n$ in cela števila $K\in\{0,\ldots,n\}$, notacija $\binom{n}{k}$ označuje:
\[
\binom{n}{k} = \frac{n!}{k!(n-k)!} \enspace .
\]
Binomski koeficient  $\binom{n}{k}$ (izgovorjeno kot ``$n$ izbere $k$'') šteje, koliko podmnožic elementa $n$ ima velikost $k$, npr. število različnih možnosti pri izbiranju $k$ različnih celih števil iz seta $\{1,\ldots,n\}$.

\subsection{Asimptotična Notacija}

\index{asymptotic notation}%
\index{big-Oh notation}%
\index{O@$O$ notation}%

Ko v knjigi analiziramo podatkovne strukture, želimo govoriti o časovnem poteku različnih operacij. Točen čas se bo seveda razlikoval od računalnika do računalnika, pa tudi od izvedbe do izvedbe na določenem računalniku. Ko govorimo o časovni zahtevnosti operacije, se nanašamo na število inštrukcij opravljenih za določeno operacijo. Tudi za enostavno kodo, je lahko to število težko za natančno določiti. Zato, bomo namesto analiziranja natančnega časovnega poteka uporabljali tako imenovano \emph{veliko-Oh notacijo}: Za funkcijo $f(n)$,
$O(f(n))$ določi set funkciji, 


\[
O(f(n)) = \left\{
\begin{array}{l}
g(n):\mbox{obstaja tak $c>0$, in $n_0$ da velja} \\
\quad\mbox{$g(n) \le c\cdot f(n)$ za vse $n\ge n_0$} 
\end{array} \right\} \enspace .
\]
Grafično mišljeno, ta set sestavljajo funkcije $g(n)$ za katere velja:
$c\cdot f(n)$ začne prevladovati $g(n)$ ko je $n$ dovolj velik.

Po navadi uporabimo asimptotično notacijo za poenostavitev funkcij. Npr. na mesto  $5n\log n + 8n – 200$, lahko zapišemo $O(n\log n)$.
To je dokazano na naslednji način:
\begin{align*} 
5n\log n + 8n - 200
& \le 5n\log n + 8n \\
& \le 5n\log n + 8n\log n & \mbox{ za $n\ge 2$ (zato da $\log n \ge 1$)}
\\
& \le 13n\log n \enspace .
\end{align*}

To dokazuja da je funkcija $f(n)=5n\log n + 8n – 200$ v množici $O(n\log n)$ z uporabo konstant $c=13$ in $n_0 = 2$.

Pri uporabi asimptotične notacije poznamo veliko bližnjic. Prva:
\[ O(n^{c_1}) \subset O(n^{c_2}) \enspace ,\]
za vsak $c_1 < c_2$. Druga: Za katerokoli konstanto $a,b,c > 0$,
\[ O(a) \subset O(\log n) \subset O(n^{b}) \subset O({c}^n) \enspace . \]
Te relacije so lahko pomnožene s katerokoli pozitivno vrednostjo, brez da bi se spremenile. Npr. če pomnožimo z $n$, dobimo:
\[ O(n) \subset O(n\log n) \subset O(n^{1+b}) \subset O(n{c}^n) \enspace . \]

Z nadaljevanjem dolge in ugledne tradicije, bomo pisali stvari npr. $f_1(n) = O(f(n))$, medtem ko s to notacjio želimo izraziti $f_1(n) \in O(f(n))$. Uporabili bomo tudi izjave kot so ``časovna zahtevnost te operacije je $O(f(n))$'', vendar pa bi izjava morala biti napisana  ``časovna zahtevnost te operacije je\emph{element} $O(f(n))$.'' Te krajšnjice se uporabljajo zgolj za to, da se izognemu nerodnemu jeziku in da lažje uporabimo asimptotično notacijo v besedilu enačb.
Zelo čuden primer tega se pojavi, ko napišemo izjavo:
\[
T(n) = 2\log n + O(1) \enspace .
\]
Bolj pravilno napisano kot
\[
T(n) \le 2\log n + [\mbox{član $O(1)$]} \enspace .
\]
Izraz $O(1)$ predstavi nov problem. Ker v tem izrazu ni nobene spremenljivke, ni čisto jasno katera spremeljivka se samovoljno povečuje. Brez konteksta, nemoremo vedeti. V zgornjem primeru, kjer je edina spremeljnivka $n$, lahko predpostavimo da bi se izraz moral prebrati kot $T(n) = 2\log n +
O(f(n))$, kjer $f(n) = 1$.



Velika-O notacija ni nova ali edinstvena računalniški znanosti. Uporabljal jo je številčni teoretik Paul Bachmann že leta 1893 saj je bila neizmerno uporabna za opis časovne zahtevnosti računalniških algoritmov.

Če upoštevamo naslednji del kode:
\javaimport{junk/Simple.snippet()}
\cppimport{ods/Simple.snippet()}
Ena izvedba te metode vključuje
\begin{itemize}
\item $1$ dodelitev (#int\, i\, =\, 0#),
\item $#n#+1$ primerjav (#i < n#),
\item #n# povečav (#i++#),
\item #n# izračun odmikov v polju (#a[i]#), and
\item #n# posrednih dodelitev (#a[i] = i#).
\end{itemize}
Zato lahko napišemo časovno zahtevnost kot
\[
T(#n#)=a + b(#n#+1) + c#n# + d#n# + e#n# \enspace , 
\]
kjer so $a$, $b$, $c$, $d$, in $e$ konstante, ki so odvisne od naprave ki izvaja kodo in predstavlja čas v katerem se zaporedno izvedejo dodelitve, primerjave, povečevalne operacije, izračuni odmikov v poljih, posredne dodelitve. Če pa izraz predstavlja časovno zahtevnost dveh vrstic kode, potem se taka analiza ne more ujemati z zapleteno kodo ali algoritmi. Časovno zahtevnost lahko poenostavimo z uporabo velike-O notacije, tako dobimo
\[
T(#n#)= O(#n#) \enspace .
\]
Tak zapis je veliko bolj kompakten in nam hkrati da veliko informacij.
To da je časovna zahtevnost v zgornjem primeru odvisna od konstante $a$, $b$, $c$, $d$, in $e$, pomeni da v splošnjem ne bo mogoče primerjati dveh časov izvedbe, da bi razločili kateri je hitrejši, brez da bi vedeli vrednosti konstant. Tudi če uspemo določiti te konstante (npr. z časovnimi testi), bi naša ugotovitev veljala samo za napravo na kateri smo izvajali teste.


Velika-O notacija daje smisel analiziranju zapletenih funkcij pri višjih stopnjah. Če imata dva algoritma enako veliko-O časovno izvedbo, potem ne moremo točno vedeti kateri je hitrejši in ni očitnega zmagovalca. En algoritem je lahko hitrejši na eni napravi, drugi pa na drugi napravi.
Če imata dva algoritma dokazljivo različno veliki-O časovni izvedbi, potem smo lahko prepričani, da bo algoritem z manjšo časovno zahtevnostjo hitrejši \emph{pri dovolj velikih vrednostih #n#}.

Kako lahko primerjamo veliko-O notacijo dveh različnih funkcij, prikazuje \figref{intro-asymptotics}, ki primerja stopnjo rasti $f_1(#n#)=15#n#$ proti $f_2(n)=2#n#\log#n#$.
Npr., da je $f_1(n)$ časovna komplesnost zapletenega linearnega časovnega algoritma in je $f_2(n)$ časovna kompleksnost bistveno preprostejšega algoritma ki temelji na vzorcu deli in vladaj. Iz tega je razvidno, da čeprav je $f_1(#n#)$ večji od $f_2(n)$ pri manjših vrednostih #n#, velja nasprotno za velike vrednosti #n#. Po določenem času bo $f_1(#n#)$ zmagal zaradi stalne povečave širine marže. Analize, ki uporabljajo veliko-O notacijo, kažejo da se bo to zgodilo, ker je $O(#n#)\subset O(#n#\log #n#)$.




\begin{figure}
\begin{center}
\newlength{\tmpa}\setlength{\tmpa}{.98\linewidth}
\addtolength{\tmpa}{-4mm}
\resizebox{\tmpa}{!}{\begin{tikzpicture}[gnuplot]
%% generated with GNUPLOT 4.4p3 (Lua 5.1.4; terminal rev. 97, script rev. 96a)
%% Mon 17 Feb 2014 12:44:43 PM CET
\gpcolor{gp lt color border}
\gpsetlinetype{gp lt border}
\gpsetlinewidth{1.00}
\draw[gp path] (1.688,0.985)--(1.868,0.985);
\draw[gp path] (12.147,0.985)--(11.967,0.985);
\node[gp node right] at (1.504,0.985) { 0};
\draw[gp path] (1.688,1.768)--(1.868,1.768);
\draw[gp path] (12.147,1.768)--(11.967,1.768);
\node[gp node right] at (1.504,1.768) { 200};
\draw[gp path] (1.688,2.552)--(1.868,2.552);
\draw[gp path] (12.147,2.552)--(11.967,2.552);
\node[gp node right] at (1.504,2.552) { 400};
\draw[gp path] (1.688,3.335)--(1.868,3.335);
\draw[gp path] (12.147,3.335)--(11.967,3.335);
\node[gp node right] at (1.504,3.335) { 600};
\draw[gp path] (1.688,4.118)--(1.868,4.118);
\draw[gp path] (12.147,4.118)--(11.967,4.118);
\node[gp node right] at (1.504,4.118) { 800};
\draw[gp path] (1.688,4.901)--(1.868,4.901);
\draw[gp path] (12.147,4.901)--(11.967,4.901);
\node[gp node right] at (1.504,4.901) { 1000};
\draw[gp path] (1.688,5.685)--(1.868,5.685);
\draw[gp path] (12.147,5.685)--(11.967,5.685);
\node[gp node right] at (1.504,5.685) { 1200};
\draw[gp path] (1.688,6.468)--(1.868,6.468);
\draw[gp path] (12.147,6.468)--(11.967,6.468);
\node[gp node right] at (1.504,6.468) { 1400};
\draw[gp path] (1.688,7.251)--(1.868,7.251);
\draw[gp path] (12.147,7.251)--(11.967,7.251);
\node[gp node right] at (1.504,7.251) { 1600};
\draw[gp path] (2.639,0.985)--(2.639,1.165);
\draw[gp path] (2.639,7.251)--(2.639,7.071);
\node[gp node center] at (2.639,0.677) { 10};
\draw[gp path] (3.695,0.985)--(3.695,1.165);
\draw[gp path] (3.695,7.251)--(3.695,7.071);
\node[gp node center] at (3.695,0.677) { 20};
\draw[gp path] (4.752,0.985)--(4.752,1.165);
\draw[gp path] (4.752,7.251)--(4.752,7.071);
\node[gp node center] at (4.752,0.677) { 30};
\draw[gp path] (5.808,0.985)--(5.808,1.165);
\draw[gp path] (5.808,7.251)--(5.808,7.071);
\node[gp node center] at (5.808,0.677) { 40};
\draw[gp path] (6.865,0.985)--(6.865,1.165);
\draw[gp path] (6.865,7.251)--(6.865,7.071);
\node[gp node center] at (6.865,0.677) { 50};
\draw[gp path] (7.921,0.985)--(7.921,1.165);
\draw[gp path] (7.921,7.251)--(7.921,7.071);
\node[gp node center] at (7.921,0.677) { 60};
\draw[gp path] (8.978,0.985)--(8.978,1.165);
\draw[gp path] (8.978,7.251)--(8.978,7.071);
\node[gp node center] at (8.978,0.677) { 70};
\draw[gp path] (10.034,0.985)--(10.034,1.165);
\draw[gp path] (10.034,7.251)--(10.034,7.071);
\node[gp node center] at (10.034,0.677) { 80};
\draw[gp path] (11.091,0.985)--(11.091,1.165);
\draw[gp path] (11.091,7.251)--(11.091,7.071);
\node[gp node center] at (11.091,0.677) { 90};
\draw[gp path] (12.147,0.985)--(12.147,1.165);
\draw[gp path] (12.147,7.251)--(12.147,7.071);
\node[gp node center] at (12.147,0.677) { 100};
\draw[gp path] (1.688,7.251)--(1.688,0.985)--(12.147,0.985)--(12.147,7.251)--cycle;
\node[gp node center,rotate=-270] at (0.246,4.118) {$f({\color{var}\mathtt{n}})$};
\node[gp node center] at (6.917,0.215) {{\color{var}\texttt{n}}};
\node[gp node right] at (10.679,1.627) {$15{\mathtt{n}}$};
\gpcolor{\gprgb{0}{376}{678}}
\gpsetlinetype{gp lt plot 0}
\gpsetlinewidth{2.00}
\draw[gp path] (10.863,1.627)--(11.779,1.627);
\draw[gp path] (1.688,1.044)--(1.794,1.102)--(1.899,1.161)--(2.005,1.220)--(2.111,1.279)%
  --(2.216,1.337)--(2.322,1.396)--(2.428,1.455)--(2.533,1.514)--(2.639,1.572)--(2.744,1.631)%
  --(2.850,1.690)--(2.956,1.749)--(3.061,1.807)--(3.167,1.866)--(3.273,1.925)--(3.378,1.984)%
  --(3.484,2.042)--(3.590,2.101)--(3.695,2.160)--(3.801,2.219)--(3.907,2.277)--(4.012,2.336)%
  --(4.118,2.395)--(4.224,2.454)--(4.329,2.512)--(4.435,2.571)--(4.540,2.630)--(4.646,2.689)%
  --(4.752,2.747)--(4.857,2.806)--(4.963,2.865)--(5.069,2.924)--(5.174,2.982)--(5.280,3.041)%
  --(5.386,3.100)--(5.491,3.159)--(5.597,3.217)--(5.703,3.276)--(5.808,3.335)--(5.914,3.393)%
  --(6.020,3.452)--(6.125,3.511)--(6.231,3.570)--(6.336,3.628)--(6.442,3.687)--(6.548,3.746)%
  --(6.653,3.805)--(6.759,3.863)--(6.865,3.922)--(6.970,3.981)--(7.076,4.040)--(7.182,4.098)%
  --(7.287,4.157)--(7.393,4.216)--(7.499,4.275)--(7.604,4.333)--(7.710,4.392)--(7.815,4.451)%
  --(7.921,4.510)--(8.027,4.568)--(8.132,4.627)--(8.238,4.686)--(8.344,4.745)--(8.449,4.803)%
  --(8.555,4.862)--(8.661,4.921)--(8.766,4.980)--(8.872,5.038)--(8.978,5.097)--(9.083,5.156)%
  --(9.189,5.215)--(9.295,5.273)--(9.400,5.332)--(9.506,5.391)--(9.611,5.450)--(9.717,5.508)%
  --(9.823,5.567)--(9.928,5.626)--(10.034,5.685)--(10.140,5.743)--(10.245,5.802)--(10.351,5.861)%
  --(10.457,5.919)--(10.562,5.978)--(10.668,6.037)--(10.774,6.096)--(10.879,6.154)--(10.985,6.213)%
  --(11.091,6.272)--(11.196,6.331)--(11.302,6.389)--(11.407,6.448)--(11.513,6.507)--(11.619,6.566)%
  --(11.724,6.624)--(11.830,6.683)--(11.936,6.742)--(12.041,6.801)--(12.147,6.859);
\gpcolor{gp lt color border}
\node[gp node right] at (10.679,1.319) {$2{\mathtt{n}}\log{\mathtt{n}}$};
\gpcolor{\gprgb{867}{94}{122}}
\gpsetlinetype{gp lt plot 1}
\draw[gp path] (10.863,1.319)--(11.779,1.319);
\draw[gp path] (1.688,0.985)--(1.794,1.001)--(1.899,1.022)--(2.005,1.048)--(2.111,1.076)%
  --(2.216,1.106)--(2.322,1.139)--(2.428,1.173)--(2.533,1.208)--(2.639,1.245)--(2.744,1.283)%
  --(2.850,1.322)--(2.956,1.362)--(3.061,1.402)--(3.167,1.444)--(3.273,1.486)--(3.378,1.529)%
  --(3.484,1.573)--(3.590,1.617)--(3.695,1.662)--(3.801,1.707)--(3.907,1.753)--(4.012,1.800)%
  --(4.118,1.847)--(4.224,1.894)--(4.329,1.942)--(4.435,1.991)--(4.540,2.039)--(4.646,2.088)%
  --(4.752,2.138)--(4.857,2.188)--(4.963,2.238)--(5.069,2.289)--(5.174,2.340)--(5.280,2.391)%
  --(5.386,2.443)--(5.491,2.495)--(5.597,2.547)--(5.703,2.600)--(5.808,2.652)--(5.914,2.705)%
  --(6.020,2.759)--(6.125,2.813)--(6.231,2.866)--(6.336,2.921)--(6.442,2.975)--(6.548,3.030)%
  --(6.653,3.085)--(6.759,3.140)--(6.865,3.195)--(6.970,3.251)--(7.076,3.307)--(7.182,3.363)%
  --(7.287,3.419)--(7.393,3.476)--(7.499,3.532)--(7.604,3.589)--(7.710,3.646)--(7.815,3.703)%
  --(7.921,3.761)--(8.027,3.819)--(8.132,3.876)--(8.238,3.934)--(8.344,3.993)--(8.449,4.051)%
  --(8.555,4.110)--(8.661,4.168)--(8.766,4.227)--(8.872,4.286)--(8.978,4.346)--(9.083,4.405)%
  --(9.189,4.464)--(9.295,4.524)--(9.400,4.584)--(9.506,4.644)--(9.611,4.704)--(9.717,4.765)%
  --(9.823,4.825)--(9.928,4.886)--(10.034,4.946)--(10.140,5.007)--(10.245,5.068)--(10.351,5.129)%
  --(10.457,5.191)--(10.562,5.252)--(10.668,5.314)--(10.774,5.375)--(10.879,5.437)--(10.985,5.499)%
  --(11.091,5.561)--(11.196,5.623)--(11.302,5.686)--(11.407,5.748)--(11.513,5.811)--(11.619,5.874)%
  --(11.724,5.936)--(11.830,5.999)--(11.936,6.062)--(12.041,6.126)--(12.147,6.189);
\gpcolor{gp lt color border}
\gpsetlinetype{gp lt border}
\gpsetlinewidth{1.00}
\draw[gp path] (1.688,7.251)--(1.688,0.985)--(12.147,0.985)--(12.147,7.251)--cycle;
%% coordinates of the plot area
\gpdefrectangularnode{gp plot 1}{\pgfpoint{1.688cm}{0.985cm}}{\pgfpoint{12.147cm}{7.251cm}}
\end{tikzpicture}
%% gnuplot variables
}\\[4ex]
\resizebox{.98\linewidth}{!}{\begin{tikzpicture}[gnuplot]
%% generated with GNUPLOT 4.4p3 (Lua 5.1.4; terminal rev. 97, script rev. 96a)
%% Mon 17 Feb 2014 12:44:43 PM CET
\gpcolor{gp lt color border}
\gpsetlinetype{gp lt border}
\gpsetlinewidth{1.00}
\draw[gp path] (2.056,0.985)--(2.236,0.985);
\draw[gp path] (12.147,0.985)--(11.967,0.985);
\node[gp node right] at (1.872,0.985) { 0};
\draw[gp path] (2.056,2.029)--(2.236,2.029);
\draw[gp path] (12.147,2.029)--(11.967,2.029);
\node[gp node right] at (1.872,2.029) { 50000};
\draw[gp path] (2.056,3.074)--(2.236,3.074);
\draw[gp path] (12.147,3.074)--(11.967,3.074);
\node[gp node right] at (1.872,3.074) { 100000};
\draw[gp path] (2.056,4.118)--(2.236,4.118);
\draw[gp path] (12.147,4.118)--(11.967,4.118);
\node[gp node right] at (1.872,4.118) { 150000};
\draw[gp path] (2.056,5.162)--(2.236,5.162);
\draw[gp path] (12.147,5.162)--(11.967,5.162);
\node[gp node right] at (1.872,5.162) { 200000};
\draw[gp path] (2.056,6.207)--(2.236,6.207);
\draw[gp path] (12.147,6.207)--(11.967,6.207);
\node[gp node right] at (1.872,6.207) { 250000};
\draw[gp path] (2.056,7.251)--(2.236,7.251);
\draw[gp path] (12.147,7.251)--(11.967,7.251);
\node[gp node right] at (1.872,7.251) { 300000};
\draw[gp path] (2.055,0.985)--(2.055,1.165);
\draw[gp path] (2.055,7.251)--(2.055,7.071);
\node[gp node center] at (2.055,0.677) { 0};
\draw[gp path] (3.064,0.985)--(3.064,1.165);
\draw[gp path] (3.064,7.251)--(3.064,7.071);
\node[gp node center] at (3.064,0.677) { 1000};
\draw[gp path] (4.073,0.985)--(4.073,1.165);
\draw[gp path] (4.073,7.251)--(4.073,7.071);
\node[gp node center] at (4.073,0.677) { 2000};
\draw[gp path] (5.083,0.985)--(5.083,1.165);
\draw[gp path] (5.083,7.251)--(5.083,7.071);
\node[gp node center] at (5.083,0.677) { 3000};
\draw[gp path] (6.092,0.985)--(6.092,1.165);
\draw[gp path] (6.092,7.251)--(6.092,7.071);
\node[gp node center] at (6.092,0.677) { 4000};
\draw[gp path] (7.101,0.985)--(7.101,1.165);
\draw[gp path] (7.101,7.251)--(7.101,7.071);
\node[gp node center] at (7.101,0.677) { 5000};
\draw[gp path] (8.110,0.985)--(8.110,1.165);
\draw[gp path] (8.110,7.251)--(8.110,7.071);
\node[gp node center] at (8.110,0.677) { 6000};
\draw[gp path] (9.119,0.985)--(9.119,1.165);
\draw[gp path] (9.119,7.251)--(9.119,7.071);
\node[gp node center] at (9.119,0.677) { 7000};
\draw[gp path] (10.129,0.985)--(10.129,1.165);
\draw[gp path] (10.129,7.251)--(10.129,7.071);
\node[gp node center] at (10.129,0.677) { 8000};
\draw[gp path] (11.138,0.985)--(11.138,1.165);
\draw[gp path] (11.138,7.251)--(11.138,7.071);
\node[gp node center] at (11.138,0.677) { 9000};
\draw[gp path] (12.147,0.985)--(12.147,1.165);
\draw[gp path] (12.147,7.251)--(12.147,7.071);
\node[gp node center] at (12.147,0.677) { 10000};
\draw[gp path] (2.056,7.251)--(2.056,0.985)--(12.147,0.985)--(12.147,7.251)--cycle;
\node[gp node center,rotate=-270] at (0.246,4.118) {$f({\color{var}\mathtt{n}})$};
\node[gp node center] at (7.101,0.215) {{\color{var}\texttt{n}}};
\node[gp node right] at (10.679,1.627) {$15{\mathtt{n}}$};
\gpcolor{\gprgb{0}{376}{678}}
\gpsetlinetype{gp lt plot 0}
\gpsetlinewidth{2.00}
\draw[gp path] (10.863,1.627)--(11.779,1.627);
\draw[gp path] (2.056,0.985)--(2.158,1.017)--(2.260,1.049)--(2.362,1.080)--(2.464,1.112)%
  --(2.566,1.144)--(2.668,1.175)--(2.770,1.207)--(2.871,1.238)--(2.973,1.270)--(3.075,1.302)%
  --(3.177,1.333)--(3.279,1.365)--(3.381,1.397)--(3.483,1.428)--(3.585,1.460)--(3.687,1.492)%
  --(3.789,1.523)--(3.891,1.555)--(3.993,1.587)--(4.095,1.618)--(4.197,1.650)--(4.298,1.681)%
  --(4.400,1.713)--(4.502,1.745)--(4.604,1.776)--(4.706,1.808)--(4.808,1.840)--(4.910,1.871)%
  --(5.012,1.903)--(5.114,1.935)--(5.216,1.966)--(5.318,1.998)--(5.420,2.030)--(5.522,2.061)%
  --(5.624,2.093)--(5.725,2.124)--(5.827,2.156)--(5.929,2.188)--(6.031,2.219)--(6.133,2.251)%
  --(6.235,2.283)--(6.337,2.314)--(6.439,2.346)--(6.541,2.378)--(6.643,2.409)--(6.745,2.441)%
  --(6.847,2.473)--(6.949,2.504)--(7.051,2.536)--(7.152,2.567)--(7.254,2.599)--(7.356,2.631)%
  --(7.458,2.662)--(7.560,2.694)--(7.662,2.726)--(7.764,2.757)--(7.866,2.789)--(7.968,2.821)%
  --(8.070,2.852)--(8.172,2.884)--(8.274,2.916)--(8.376,2.947)--(8.478,2.979)--(8.579,3.010)%
  --(8.681,3.042)--(8.783,3.074)--(8.885,3.105)--(8.987,3.137)--(9.089,3.169)--(9.191,3.200)%
  --(9.293,3.232)--(9.395,3.264)--(9.497,3.295)--(9.599,3.327)--(9.701,3.359)--(9.803,3.390)%
  --(9.905,3.422)--(10.006,3.453)--(10.108,3.485)--(10.210,3.517)--(10.312,3.548)--(10.414,3.580)%
  --(10.516,3.612)--(10.618,3.643)--(10.720,3.675)--(10.822,3.707)--(10.924,3.738)--(11.026,3.770)%
  --(11.128,3.802)--(11.230,3.833)--(11.332,3.865)--(11.433,3.896)--(11.535,3.928)--(11.637,3.960)%
  --(11.739,3.991)--(11.841,4.023)--(11.943,4.055)--(12.045,4.086)--(12.147,4.118);
\gpcolor{gp lt color border}
\node[gp node right] at (10.679,1.319) {$2{\mathtt{n}}\log{\mathtt{n}}$};
\gpcolor{\gprgb{867}{94}{122}}
\gpsetlinetype{gp lt plot 1}
\draw[gp path] (10.863,1.319)--(11.779,1.319);
\draw[gp path] (2.056,0.985)--(2.158,1.013)--(2.260,1.050)--(2.362,1.090)--(2.464,1.132)%
  --(2.566,1.175)--(2.668,1.219)--(2.770,1.265)--(2.871,1.311)--(2.973,1.359)--(3.075,1.407)%
  --(3.177,1.455)--(3.279,1.504)--(3.381,1.554)--(3.483,1.604)--(3.585,1.654)--(3.687,1.705)%
  --(3.789,1.756)--(3.891,1.808)--(3.993,1.860)--(4.095,1.912)--(4.197,1.965)--(4.298,2.017)%
  --(4.400,2.071)--(4.502,2.124)--(4.604,2.178)--(4.706,2.232)--(4.808,2.286)--(4.910,2.340)%
  --(5.012,2.395)--(5.114,2.449)--(5.216,2.504)--(5.318,2.560)--(5.420,2.615)--(5.522,2.670)%
  --(5.624,2.726)--(5.725,2.782)--(5.827,2.838)--(5.929,2.894)--(6.031,2.951)--(6.133,3.007)%
  --(6.235,3.064)--(6.337,3.121)--(6.439,3.178)--(6.541,3.235)--(6.643,3.292)--(6.745,3.350)%
  --(6.847,3.407)--(6.949,3.465)--(7.051,3.523)--(7.152,3.581)--(7.254,3.639)--(7.356,3.697)%
  --(7.458,3.755)--(7.560,3.814)--(7.662,3.872)--(7.764,3.931)--(7.866,3.990)--(7.968,4.048)%
  --(8.070,4.107)--(8.172,4.166)--(8.274,4.226)--(8.376,4.285)--(8.478,4.344)--(8.579,4.404)%
  --(8.681,4.463)--(8.783,4.523)--(8.885,4.582)--(8.987,4.642)--(9.089,4.702)--(9.191,4.762)%
  --(9.293,4.822)--(9.395,4.882)--(9.497,4.943)--(9.599,5.003)--(9.701,5.063)--(9.803,5.124)%
  --(9.905,5.185)--(10.006,5.245)--(10.108,5.306)--(10.210,5.367)--(10.312,5.428)--(10.414,5.489)%
  --(10.516,5.550)--(10.618,5.611)--(10.720,5.672)--(10.822,5.733)--(10.924,5.795)--(11.026,5.856)%
  --(11.128,5.917)--(11.230,5.979)--(11.332,6.041)--(11.433,6.102)--(11.535,6.164)--(11.637,6.226)%
  --(11.739,6.288)--(11.841,6.350)--(11.943,6.412)--(12.045,6.474)--(12.147,6.536);
\gpcolor{gp lt color border}
\gpsetlinetype{gp lt border}
\gpsetlinewidth{1.00}
\draw[gp path] (2.056,7.251)--(2.056,0.985)--(12.147,0.985)--(12.147,7.251)--cycle;
%% coordinates of the plot area
\gpdefrectangularnode{gp plot 1}{\pgfpoint{2.056cm}{0.985cm}}{\pgfpoint{12.147cm}{7.251cm}}
\end{tikzpicture}
%% gnuplot variables
}
\end{center}
\caption{Plots of $15#n#$ versus $2#n#\log#n#$.}
\figlabel{intro-asymptotics}
\end{figure}

V nekaterih primerih bomo uporabili asimptotično notacijo na funkcijah z več kot eno spremenljivko. Predpisan ni noben standard, ampak za naš namen, je naslednja definicija zadovoljiva:

\[
O(f(n_1,\ldots,n_k)) = 
\left\{\begin{array}{@{}l@{}}
g(n_1,\ldots,n_k):\mbox{obstaja $c>0$, in $z$ da velja} \\
\quad \mbox{$g(n_1,\ldots,n_k) \le c\cdot f(n_1,\ldots,n_k)$} \\
\qquad \mbox{za vse $n_1,\ldots,n_k$ da velja $g(n_1,\ldots,n_k)\ge z$} 
\end{array}\right\} \enspace .
\]
Ta definicija zajema položaj, ki nas zanima: ko $g$ prevzame višje vrednosti zaradi argumenta $n_1,\ldots,n_k$. Ta definicija se sklada z univarijatno definicijo $O(f(n))$ ko je $f(n)$ naraščujoča funkcija $n$. Bralci naj bodo pozorni, da je lahko v drugih besedilih uporabljena asimptotična notacija drugače.

\subsection{Naključnost in verjetnost}
\seclabel{randomization}

\index{randomization}%
\index{probability}%
\index{randomized data structure}%
\index{randomized algorithm}%

Nekatere podatkovne strukture predstavljene v knjigi so \emph{naključne}; odločajo se naključno in neodvisno od podatkov ki so spravljeni v njih in od opraciji ki se izvajajo nad njimi. Zaradi tega, se lahko časi izvajanja razlukujejo med seboj, kljub temu da uporabimo enako zaporedje operacij nad strukturo. Ko analiziramo podatkovne strukture, nas zanima povprečje, oziroma \emph{pričakovan} čas poteka.

\index{expected running time}%
\index{running time!expected}%







Formalno, je čas poteka operacije na naključni podatkovni strukturi, naključna spremenljivka, želimo pa preučevati njeno \emph{pričakovano vrednost}.

\index{expected value}%
Za diskretno naključno spremenljivko $X$, ki zavzame vrednosti neke univerzalne mnozice $U$, je pričakovana vrednost $X$ označena z $\E[X]$ podana z enačbo
\[
\E[X] = \sum_{x\in U} x\cdot\Pr\{X=x\} \enspace .
\]

Tukaj $\Pr\{\mathcal{E}\}$ označuje verjetnost da se pojavi dogodek $\mathcal{E}$.
V vseh primerih v knjigi, so te verjetnosti v spoštovanju z naključnimi odločitvami narejenimi s strani podatkovnih struktur; ne moremo sklepati, da so naključni podatki, ki so shranjeni v strukturi, niti sekvence operaciji izvedene na podatkovni strukturi.

Ena pomembnejših lastnosti pričakovane verjetnosti je \emph{linearnost pričakovanja}.

\index{linearity of expectation}%
Za katerekoli dve naključne spremenljivke $X$ in $Y$,
\[
\E[X+Y] = \E[X] + \E[Y] \enspace .
\]
Bolj splošno, za katerokoli naključno spremenljivko $X_1,\ldots,X_k$,
\[
\E\left[\sum_{i=1}^k X_k\right] = \sum_{i=1}^k \E[X_i] \enspace .
\]
Linearnost pričakovanja nam dovoljuje da razbijemo zapletene naključne spremenljivke(kot leva stran od zgornjih enačb), v vsote enostavnejših naključnih spremenljivk(desna stran).

Uporaben trik, ki ga bomo pogosto uporabljali, je definiranje indikatorja naključnih \emph{spremenljivk}.
\index{indicator random variable}%
Te binarne spremenljivke so uporabne ko želimo nekaj šteti in so najboljše ponazorjene s primerom.
Npr. da vržemo pravičen kovanec $k$ krat in želimo vedeti pričakovano število, koliko krat bo kovanec kazal glavo. 

\index{coin toss}%
Intuitivno vemo da je odgovor $k/2$. Če pa želimo to dokazati z definicijo pričakovane vrednosti, dobimo
\begin{align*}
\E[X] & = \sum_{i=0}^k i\cdot\Pr\{X=i\} \\
& = \sum_{i=0}^k i\cdot\binom{k}{i}/2^k \\
& = k\cdot \sum_{i=0}^{k-1}\binom{k-1}{i}/2^k \\
& = k/2 \enspace .
\end{align*}
To zahteva, da vemo dovolj da izračunamo, da $\Pr\{X=i\}
= \binom{k}{i}/2^k$, in da vemo binomske identitete
$i\binom{k}{i}=k\binom{k-1}{i}$ and $\sum_{i=0}^{k} \binom{k}{i} = 2^{k}$.



Z uporabo indikatorskih spremenljivk in linearnostjo pričakovanja so stvari veliko lažje. Za vsak $i\in\{1,\ldots,k\}$, opredelimo indikatorsko naključno spremenljivko.
\[
I_i = \begin{cases}
1 & \text{če je  $i$ti met kovanca glava} \\
0 & \text{drugače.}
\end{cases}
\]
Potem
\[ \E[I_i] = (1/2)1 + (1/2)0 = 1/2 \enspace . \]
Sedaj, $X=\sum_{i=1}^k I_i$, so
\begin{align*}
\E[X] & = \E\left[\sum_{i=1}^k I_i\right] \\
& = \sum_{i=1}^k \E[I_i] \\
& = \sum_{i=1}^k 1/2 \\
& = k/2 \enspace .
\end{align*}
To je malo bolj zapleteno, vendar za to ne potrebujemo nobenih magičnih identitet ali računanja kakršnih koli ne trivijalnih verjetnosti. Še boljše, strinja se z intuicijo, da pričakujemo polovico kovancev, da pristanejo na glavi, točno zato ker vsak posamezni kovanec pristane na glavi z verjetnostjo $1/2$.

\section{Vmesniki}
\seclabel{interfaces}
Pri razpravi o podatkovnih strukturah je pomembno poznati razliko med vmesnikom podatkovne strukture in njegovo implementacijo. Vmesnik opisuje kaj podatkovna struktura počne, medtem ko implementacija opisuje kako to počne.

\emph{Vmesnik},
\index{interface}%
\index{abstract data type|see{interface}}%
včasih imenovan tudi  \emph{abstrakten podatkovni tip}, definira množico operacij, ki so podprte s strani podatkovne strukture in semantiko oziroma pomenom teh operacij. Vmesnik nam ne pove nič o tem, kako podatkovna struktura implementira te operacije. Pove nam samo katere operacije so podprte, vključno z specifikacijami o vrstah argumentov, ki jih vsaka operacija sprejme in vrednostmi ki jih operacije vračajo.

\emph{implementacija} podatkovne strukture, po drugi strani vsebuje notranjo predstavitev podatkovne strukture, vključno z definicijami algoritmov, ki implementirajo operacije podprte s strani podatkovne strukture. Zatorej imamo lahko veliko implementaciji enega samega vmesnika. Na primer v \chapref{arrays}, bomo videli implementacije vmesnika #seznama# z uporabo polj in v 
\chapref{linkedlists} bomo videli implementacije vmesnikov #seznama# z uporabo podatkovnih struktur, katere uporabljajo kazalce. Obe implementirajo isti vmesnik, #seznam#, vendar na drugačen način.


\subsection{#Vrsta#, #Sklad#, in #Deque# Vmesniki}

Vmesnik #Vrste# predstavlja zbirko elementov med katere lahko dodamo ali izbrisemo naslednji element. Bolj natančno, operaceije podprte z vmesnikom #vrste# so
\begin{itemize}
\item #add(x)#: dodaj vrednost #x#  #vrsti#
\item #remove()#: izbriši naslednjo (prej dodano) vrednost, #y#, iz #vrste# in vrni #y#
\end{itemize}
Opazimo lahko da metoda #remove()# ne sprejme nobenega argumenta. Implementacija #vrste# odloča kateri element bo izbrisan iz vrste. Poznamo veliko implementaciji vrste, najbolj pogoste pa so FIFO, vrste s prednostjo in LIFO.
\emph{FIFO (first-in-first-out) #vrsta#},
\index{FIFO queue}%
\index{queue!FIFO}%
ki je narisana v \figref{queue}, odstrani elemente v enakem vrstnem redu kot so bili dodani, enako kot vrsta deluje, ko stojimo v vrsti za na blagajno v trgovini. To je najbolj pogosta implementaicija #vrste#, zato je kvalifikant FIFO pogosto izpuščen. V drugih besedilih, se #add(x)# in #remove()# operacije na #vrsti# FIFO pogosto ključejo #enqueue(x)# oziroma #dequeue(x)#

\begin{figure}
\centering{\includegraphics[width=\ScaleIfNeeded]{figs/queue}}
\caption[A FIFO queue]{FIFO #vrsta#.}
\figlabel{queue}
\end{figure}

\emph{#Vrste# s prednostjo},
\index{priority queue}%
\index{priority queue|seealso{heap}}%
\index{queue!priority}%
prikazane na \figref{prioqueue}, vedno odstranijo najmanjši element iz #vrste#. To je podobno sistemu sprejema bolnikov v bolnicah. Ob prihodu zdravniki ocenijo poškodbo/bolezen bolnika in ga napotijo v čakalno sobo. Ko je zdravnik na voljo, prvo zdravi bolnika z najbolj smrtno nevarno poškodbo/boleznijo. V drugih besedilih je #remove()# operacija na #vrsti# s prednostjo ponavadi imenovana #deleteMin()#. 

\begin{figure}
\centering{\includegraphics[width=\ScaleIfNeeded]{figs/prioqueue}}
\caption[A priority queue]{#Vrsta# s prednostjo.}
\figlabel{prioqueue}
\end{figure}

%This is similar to the way
%many airlines manage upgrades to the business class on their flights.
%When a business-class seat becomes available it is given to the most
%important customer waiting on an upgrade.

Zelo pogosta implementacija  vrste je LIFO (last-in-first-out)
\index{LIFO queue}%
\index{LIFO queue|seealso{stack}}%
\index{queue!LIFO}%
\index{stack}%
prikazana na \figref{stack}. Na \emph{LIFO vrsti} je izbrisan nazadnje dodan element. To je najboljse prikazano s kupom krožnikov; krožniki so postavljeni na vrh kupa, pravtako so odstranjeni iz vrha kupa. Ta struktura je tako pogosta, da je dobila svoje ime: #sklad#. Pogosto ko govorimo o #skladu#, so imena #add(x)# in #remove()# spremenjena v #push(x)# in #pop()#; S tem se izognemo zamenjavi implementacij vrst LIFO in FIFO.



\begin{figure}
\centering{\includegraphics[width=\ScaleIfNeeded]{figs/stack}}
\caption[A stack]{sklad.}
\figlabel{stack}
\end{figure}


#Deque#
\index{deque}%
je generalizacija obeh FIFO #vrste# in LIFO #vrste# (#sklad#). #Deque# predstavlja sekvenco elementov, z začetkom in koncem. Elementi so lahko dodani na začetek ali pa na konec. Imena #deque# so samoumevna: #addFirst(x)#, #removeFirst()#, #addLast(x)#,
in #removeLast()#. #Sklad# je lahko implementiran samo z uporabo  #addFirst(x)# in #removeFirst()#, medtem ko FIFO #vrsta# je lahko implementirana z uporabo #addLast(x)# in #removeFirst()#.

\subsection{Vmesnik #seznama#: linearne sekvence}

Ta knjiga govori zelo malo o FIFO #vrsti#, #skladu# ali #deque# vmesnikih, ker so vmesniki vključeni z vmesnikom #seznama#. #Seznam#, \index{List@#List#}% prikazanega na \figref{list}, predstavlja sekvence $#x#_0,\ldots,#x#_{#n#-1}$, vrednosti. Vmesnik #seznama# vključuje naslednje operacije:
\begin{enumerate}
\item #size()#: vrne #n#, dolžino seznama
\item #get(i)#: vrne vrednost $#x#_{#i#}$
\item #set(i,x)#: nastavi vrednost $#x#_{#i#}$ na #x#
\item #add(i,x)#: doda #x# na mesto #i#, izrine
$#x#_{#i#},\ldots,#x#_{#n#-1}$; \\ 
Nastavi $#x#_{j+1}=#x#_j$, za vse
$j\in\{#n#-1,\ldots,#i#\}$, poveča #n#, in nastavi $#x#_i=#x#$
\item #remove(i)#izbriše vrednost $#x#_{#i#}$, izrine
$#x#_{#i+1#},\ldots,#x#_{#n#-1}$; \\ 
Nastavi $#x#_{j}=#x#_{j+1}$, za vse
$j\in\{#i#,\ldots,#n#-2\}$ in zniža #n#
\end{enumerate}

Opazimo lahko da te operacije enostavno lahko implementirajo
#deque# vmesnik:

\begin{eqnarray*}
#addFirst(x)# &\Rightarrow& #add(0,x)# \\
#removeFirst()# &\Rightarrow& #remove(0)# \\
#addLast(x)# &\Rightarrow& #add(size(),x)# \\
#removeLast()# &\Rightarrow& #remove(size()-1)#
\end{eqnarray*}

\begin{figure}
\centering{\includegraphics[width=\ScaleIfNeeded]{figs/list}}
\caption[A List]{#Seznam# predstavlja sekvenco indeksov $0,1,2,\ldots,#n#-1$. V tem #seznamu#, bi klic #get(2)# vrnil
vrednost $c$.}
\figlabel{list}
\end{figure}




Čeprav ne bomo razpravljali o vmesnikih #sklada#, #deque# in FIFO #vrste# v podpoglavjih, sta izraza #sklad# in #deque#  včasih uporabljena kot imeni podatkovnih struktur, ki implementirajo vmesnik #seznama#. V tem primeru, zelimo poudariti, da lahko te podatkovne strukture uporabimo za implementacijo vmesnika #sklada# in #deque# zelo efektivno. Na primer, #ArrayDeque# razred je implementacije vmesnika #seznama# ki implementira vse #deque# operacije v konstantnem času na operacijo.

\subsection{#USet# vmesnik: Neurejen set.}

#USet#
\index{USet@#USet#}%
vmesnik predstavlja neurejen set edinstvenih elementov, ki posnemajo matematični \emph{set}.
#Uset# vsebuje #n# \emph{različnih} elementov; noben element se ne pojavi več kot enkrat; elementi niso v nobenem določenem zaporedju. #USet# podpira naslednje operacije:

\begin{enumerate}
\item #size()#:vrne število, #n#, elementov v setu
\item #add(x)#: doda element #x# v set, če ta že ni prisoten; \\
Dodaj #x# setu, če ne obstaja tak element #y# v setu, da velja da je #x# enak #y#. Vrni #true#
Če je bil #x# dodan v set,  #false# otherwise.
\item #remove(x)#: odstrani #x# iz seta; \\
Najdi element #y# v setu, da velja da je #x# enak
#y# in odstrani #y#. Vrni #y#, ali #null# če tak element ne obstaja.
\item #find(x)#: najde #x# v setu, če obstaja; \\
Najdi element #y# v setu, da velja da je #y# enak
#x#. Vrni #y#, ali #null# če tak element ne obstaja.
\end{enumerate}


Te definicije se razlikujejo za razpoznavni element #x#, element ki ga bomo odstranili ali najdli, od elementa #y#, element ki ga bomo verjetno odstranili ali najdli.
To je zato, ker sta #x# in #y# lahko različna objekta ki sta lahko tretirana kot enaka. \javaonly{\footnote{V Javi, je to storjeno z prepisom razredovih #equals(y)# in #hashCode()# metod.}}. Tako razlikovanje je uporabno, ker dovoljuje kreiranje \emph{imenikov} ali \emph{map}, ki preslika ključe v vrednosti.
\index{dictionary}%
\index{map}%


Da naredimo imenik, eden tvori skupino objektov imenovanih #pari#, 
\index{pair}%
kateri vsebujejo \emph{ključ} in a \emph{vrednost}.
Dva #para# sta si enakovredna, če so njuni ključi enaki. Če spravimo nek par $(#k#,#v#)$ v #USet#
 in kasneje kličemo #find(x)# metodo z uporabo para $#x#=(#k#,#null#)$m bi rezultat $#y#=(#k#,#v#)$. Z drugimi besedami povedano, možno je dobiti vrednost #v#, če podamo samo ključ #k#.




\subsection{#SSet# vmesnik: Urejeni seti}
\seclabel{sset}

\index{SSet@#SSet#}%
#SSet# vmesnik predstavlja urejen set elementov. #SSet# hrani elemente v nekem zaporedju, tako da sta lahko katera koli elementa #x# in #y# primerjana med sabo. V primeru, bo to storjeno z metodo imenovano #compare(x,y)# v kateri
\[
#compare(x,y)# 
\begin{cases}
{}<0 & \text{if $#x#<#y#$} \\
{}>0 & \text{if $#x#>#y#$} \\
{}=0 & \text{if $#x#=#y#$}
\end{cases}
\]

\index{compare@#compare(x,y)#}%
#SSet# podpira #size()#, #add(x)#, in #remove(x)# metode z točno enako semantiko kot #USet#  vmesnik. Razlika med #USet# in #SSet# v metodi #find(x)#:

\begin{enumerate}
\setcounter{enumi}{3}
\item #find(x)#: locira #x# v urejenem setu; \\
Najde najmanjši element #y# v setu, da velja $#y# \ge #x#$.
Vrne #y# ali #null# če tak element ne obstaja
\end{enumerate}

Taka verzija metode #find(x)# je imenovana \emph{iskanje naslednika}..
\index{successor search}%
Temeljno se razlikuje od #USet.find(x)# saj vrne smiselen rezultat tudi če v setu ni elementa ki je enak #x#.

Razlika med #USet# in #SSet# #find(x)# operacijo je zelo pomembna in velikokrat prezrta. Dodatna funkcijonalnost priskrbljena s strani #SSet# ponavadi pride s ceno, da metoda porabi več časa za iskanje in večjo kompleknostjo kode. Na primer, večina implementacij #SSeta# omenjenih v tej knjigi imajo #find(x)# operacije, ki potrebujejo logaritmičen čas glede na velikost podatkov.
Na drugi strani, implementacija #USeta# kot #ChainedHashTable# v \chapref{hashing} ima #find(x)# operacijo ki potrebuje konstanten pričakovani čas. Ko izbiramo katero od teh struktur bomo uporabili, bi vedno morali uporabiti #USet#, razen če je dodatna funkcionalnost ki jo ponudi #SSet# nujna.

