\chapter{Dvojiška drevesa}
\chaplabel{binarytrees}
\translatedby{Erik Šimer, Vesna Horvat}{sl}

To poglavje vpeljuje eno izmed najbolj temeljnih struktur v računalništvu: dvojiška drevesa. Uporaba besede \emph{drevo}
\index{tree}%
\index{tree!binary}%
\index{binary tree}%
prihaja iz dejstva da je, ko jih rišemo, končna risba podobna drevesom iz gozda.
Obstaja veliko načinov definiranja binarnih dreves.
 Matematično je  \emph{binarno drevo} povezan,
neusmerjen, končni graf brez ciklov, katerih stopnja bi bila večja od tri.

Za večino aplikacij v računalništvu, so binarna drevesa \emph{zakoreninjena:}
\index{tree!rooted}%
\index{rooted tree}%
Posebno vozlišče #r#, v največ drugi stopnji, se imenuje \emph{koren}
drevesa. Za vsako vozlišče $#u#\neq #r#$, se drugo vozlišče na
poti od #u# do #r# imenuje \emph{starš} od #u# .
\index{parent}%
Vsako drugo vozlišče, ki meji na #u# imenujemo \emph{otrok}
\index{child} od #u#. Večina
dvojiških dreves, ki nas zanimajo, je \emph{urejena}
\index{ordered tree}%
\index{tree!ordered}%
 in tako lahko ločimo
med \emph{levi otrok} in  \emph{desni otrok} od #u#.
\index{left child}%
\index{child!left}%
\index{right child}%
\index{child!right}%

V ilustracijah, so dvojiška drevesa običajno narisana iz korena
navzdol, s korenom na vrhu slike in levim ter desnim otrokom tako,
da je levi otrok na levi strani, desni pa na desni strani
(\figref{bintree-orientation}). Na primer \figref{binary-tree}. kaže
binarno drevo z devetimi vozlišči.

\begin{figure}
  \begin{center}
    \includegraphics[scale=0.90909]{figs/bintree-traverse-1} 
  \end{center}
  \caption[Parent, left child, and right child]{Starš, levi otrok, desni otrok vozlišča  #u#
   v #BinaryTree#.}
  \figlabel{bintree-orientation}
\end{figure}


\begin{figure}
  \begin{center}
    \begin{tabular}{cc}
      \includegraphics[width=\HalfScaleIfNeeded]{figs/bintree-1} &
      \includegraphics[width=\HalfScaleIfNeeded]{figs/bintree-2} \\
      (a) & (b)
    \end{tabular}
  \end{center}
  \caption{Binarno drevo (a)~devet vozlišč in (b)~deset zunanjih vozlišč.}
  \figlabel{binary-tree}
\end{figure}

Ker so dvojiška drevesa tako pomembna, so za njih razvili določeno terminologijo:
 \emph{globina}
\index{depth}%
vozlišča, #u#, je v binarnem drevesu
dolžina poti od #u# do korena drevesa. Če je vozlišče #w#,
na poti od #u# do #r#, potem #w# imenujemo \emph{prednik}
\index{ancestor}%
od #u# in #u# pa imenujemo \emph{potomec}
\index{descendant}%
od #w#. \emph{poddrevo} od
vozlišča #u# je binarno drevo, ki ima korenine v #u#  in vsebuje vse
potomce od #u#. \emph{višina}
\index{height!in a tree} vozlišča #u#, je dolžina
najdaljše poti od #u# do enega od njenih potomcev. \emph{višina}
\index{height!of a tree}%
drevesa je višina njegovega korena.
Vozlišče #u#, je \emph{list}
\index{leaf}%
če nima nobenega otroka.

Včasih mislimo, da so drevesa utrjena z \emph{zunanjimi
vozlišči}. Vsako vozlišče, ki nima levega otroka ima zunanje
vozlišče kot za svojega levega otroka in podobno vsako vozlišče, ki
nima desnega otroka ima zunanje vozlišče kot za svojega desnega otroka (glej \figref{binary-tree}.b). Z indukcijo lahko enostavno preverimo, da ima binarno drevo z $#n#\ge 1$ pravimi vozlišči $#n#+1$ zunanjih vozlišč.


\section{#BinaryTree#: Osnovno Binarno Drevo}

\index{BinaryTree@#BinaryTree#}%
Najenostavnejši način predstavitve vozlišča #u# v binarnem drevesu je izrecno shranjevanje (največ treh) sosedov od #u#:
\javaimport{ods/BinaryTree.BTNode<Node}
\cppimport{ods/BinaryTree.BTNode}
Če kateri od treh sosedov ni prisoten, ga nastavimo na #nil#.
Na ta način sta obe zunanji vozlišči drevesa in starš korena
vrednosti #nil#.

Binarno drevo se lahko zastopa kot
\javaonly{referenca}\cpponly{pointer} do svojega korenskega vozlišča #r#:
\codeimport{ods/BinaryTree.r}

Globino vozlišča #u# lahko izračunamo tako, da štejemo korake od #u# do korenskega vozlišča:
\codeimport{ods/BinaryTree.depth(u)}


\subsection{Rekurzivni algoritmi}

\index{recursive algorithm}%
Z uporabo rekurzivnih algoritmov je izračun o binarnih drevesih enostaven.
 Na primer, za izračun velikosti (število vozlišč)
binarnega drevesa, ki je zakoreninjen v vozlišču #u#, naredimo tako da rekurzivno izračunamo velikost
dveh poddreves, ki so zakoreninjena na otroke od #u#, nato povzamemo te velikosti, in dodamo eno:

\codeimport{ods/BinaryTree.size(u)}

Za izračun višine vozlišča #u# moremo izračunati višino #u#- jevih dveh poddreves,
vzeti največjega  in mu dodati:

\codeimport{ods/BinaryTree.height(u)}

\subsection{Obiskovanje Binarnega drevesa}
\seclabel{bintree:traversal}

\index{traversal!of a binary tree}%
\index{tree traversal}%
\index{binary-tree traversal}%
Prejšnja algoritma iz prejšnjega odseka uporabljata rekurzijo, za obisk
vseh vozlišč v binarnem drevesu. Vsak od njih obišče vozlišča
binarnega drevesa v istem vrstnem redu kot naslednja koda:
\codeimport{ods/BinaryTree.traverse(u)}

Z uporabo rekurzije, lahko na ta način proizvajamo zelo kratko in preprosto kodo, lahko pa je taka koda zelo problematična.
Največja globina rekurzijska je podana z
največjo globino vozlišča v dvojiškem drevesu, tj, višina drevesa.
Če je višina drevesa zelo velika, potem lahko taka rekurzija 
porabi veliko več pomnilnika na skladu, kot ga je na voljo.

Za obhod binarnega drevesa brez rekurzije, lahko uporabimo algoritem, ki se 
zanaša na to, da ve iz kje je prišel in kam bo odšel. Glej
\figref{bintree-traverse}.  Če pridemo do vozlišča #u# od #u.parent#,
potem obiščemo #u.left#. Če pridemo do #u#
od #u.left#, potem obiščemo #u.right#. Če prispemo
 na #u# iz #u.right# , potem smo končali z obiskovanjem #u# -jevih poddreves,
in se tako vrnemo na #u.parent#. Naslednja koda izvaja to
idejo, ki vključuje ravnanje v primerih, ko katera koli od #u.left#,
#u.right# ali #u.parent# je #nil#:
\codeimport{ods/BinaryTree.traverse2()}

\begin{figure}
  \begin{center}
    \begin{tabular}{cc}
      \includegraphics[scale=0.90909]{figs/bintree-traverse-2}
      \includegraphics[scale=0.90909]{figs/bintree-3}
    \end{tabular}
  \end{center}
  \caption[Traversing a BinaryTree]{Trije primeri, ki se pojavijo na vozlišču
    #u# kadar obhodimo binarna drevesa,ki niso rekurzivna}
  \figlabel{bintree-traverse}
\end{figure}

Enake primere, ki jih lahko izračunamo z rekurzivnimi algoritmi, lahko izračunamo z iterativnimi algoritmi. 
Na primer, za izračun velikosti drevesa hranimo števec #n#, in nižamo #n# vsakič ko obiščemo novo vozlišče.
\codeimport{ods/BinaryTree.size2()}

V nekaterih implementacijah binarnih dreves, se #parent# ne uporablja.
V takih primerih, lahko še vedno uporabimo iterativno izvedbo, vendar
mora taka izvedba uporabljati #List# (ali #Stack#), saj bi tako lahko
spremljali pot od trenutnega vozlišča do korena.

Posebna vrsta prečkanja, ki ne ustreza vzorcu zgoraj navedene funkcije je
 \emph{prvi-v-širino}.
\index{breadth-first traversal}%
\index{traversal!breadth-first}%
V prvi-v-širino obhodu, so vozlišča obiskana stopnja-postopnjo,
 pri kateremu začnemo v korenu in nadaljujemo navzdol,
 kjer obiskujemo vsako vozlišče od levega proti desni (glej
\figref{bintree-bfs}). 
To je podobno načinu branja strani v Angleškem jeziku.
Prvi-v-širino obhod je implementiran z uporabo vrste #q#, ki na začetku vsebuje samo koren  #r#.
Na vsakem koraku, vzamemo naslednje vozlišče #u# iz #q# , nato procesiramo #u#, in dodamo #u.left#
in #u.right# (če niso #nil#) v #q#:
\codeimport{ods/BinaryTree.bfTraverse()}

\begin{figure}
  \begin{center}
    \includegraphics[scale=0.90909]{figs/bintree-4}
  \end{center}
  \caption{Med prvi-v-širino obhodu,  so vozlišča v binarnem drevesu obiskana po principu stopnja-po-stopnjo in levo-proti-desni za vsako stopnjo.}
  \figlabel{bintree-bfs}
\end{figure}



\translatedby{Matej Brlec}{sl}

\section{#BinarySearchTree#: Neuravnoteženo binarno iskalno drevo}
\seclabel{binarysearchtree}

\index{BinarySearchTree@#BinarySearchTree#}%
\index{binary search tree}%
\index{binary tree!search}%
#BinarySearchTree# je posebna oblika binarnega drevesa, pri katerem vsako vozlišče #u# hrani tudi podatek #u.x# iz nekega skupnega vrstnega reda. Podatki binarnega iskalnega drevesa upoštevajo \emph{lastnost binarnih iskalnih dreves}:
\index{binary search tree property}%
Za vozlišče #u# velja, da vsak podatek shranjen v poddrevesu #u.left# je manjši od #u.x# ter vsak podatek shranjen v poddrevesu #u.right# je večji od #u.x#.  Primer #BinarySearchTree# je prikazan v \figref{bst}.

\begin{figure}
  \begin{center}
    \includegraphics[scale=0.90909]{figs/bst-example}
    %\includegraphics[scale=0.90909]{figs/binary-tree-4}
  \end{center}
  \caption{Binarno iskalno drevo.}
  \figlabel{bst}
\end{figure}


\subsection{Iskanje}

\index{search path!in a binary search tree}%
Lastnost binarnega iskalnega drevesa je zelo uporabna, ker nam omogoča hitro iskanje vrednosti #x# v binarnem iskalnem drevesu. To naredimo tako, da začnemo z iskanjem vrednosti #x# v korenu #r#. Ko pregledamo vozlišče #u#, imamo tri možnosti:
\begin{enumerate}
\item Če je $#x#< #u.x#$, nadaljujemo z iskanjem v #u.left#;
\item Če je $#x#> #u.x#$, nadaljujemo z iskanjem v #u.right#;
\item Če je $#x#= #u.x#$, pomeni, da smo našli vozlišče #u#, ki hrani #x#.
\end{enumerate}
Iskanje se zaključi, ko dosežemo Možnost~3 ali ko je #u=nil# (prazen).  V prvem primeru smo našli #x#. V drugem pa sklenemo, da #x# ni v binarnem iskalnem drevesu.
\codeimport{ods/BinarySearchTree.findEQ(x)}

V \figref{bst-search} sta prikazana dva primera iskanj v binarnem iskalnem drevesu. Drugi primer prikazuje, da tudi če ne najdemo #x# v drevesu, vseeno pridobimo nekaj pomembnih informacij. Če pogledamo zadnje vozlišče #u# pri katerem se je zgodila Možnost~1, vidimo, da je #u.x# najmanjša vrednost v drevesu, ki je večja od #x#.  Podobno, zadnje vozlišče kjer se je zgodila Možnost~2 hrani največjo vrednost v drevesu, ki je manjša od #x#.  Torej, ob spremljanju zadnjega vozlišča #z# pri katerem se je zgodila Možnost~1, lahko #BinarySearchTree# implementira #find(x)# funkcijo, ki vrne najmanjšo vrednost v drevesu, ki je večja ali enaka #x#:
\codeimport{ods/BinarySearchTree.find(x)}

\begin{figure}
  \begin{center}
    \begin{tabular}{cc}
    \includegraphics[width=\HalfScaleIfNeeded]{figs/bst-example-2} &
    \includegraphics[width=\HalfScaleIfNeeded]{figs/bst-example-3} \\
    (a) & (b)
    \end{tabular}
  \end{center}
  \caption{Primer (a)~uspešnega iskanja (za $6$) ter (b)~neuspešnega iskanja (za $10$) v binarnem iskalnem drevesu.}
  \figlabel{bst-search}
\end{figure}


\subsection{Vstavljanje}

Pri vstavljanju nove vrednosti #x# v #BinarySearchTree#, najprej poiščemo #x# v drevesu. Če ga najdemo, potem vstavljanje ni potrebno. V nasprotnem primeru shranimo #x# v otroka zadnjega vozlišča #p#, ki smo ga obiskali med iskanjem za vrednostjo #x#. Ali je novo vozlišče levi ali desni otrok vozlišča #p#, je odvisno od rezultata primerjave med #x# ter #p.x#.
\codeimport{ods/BinarySearchTree.add(x)}
\codeimport{ods/BinarySearchTree.findLast(x)}
\codeimport{ods/BinarySearchTree.addChild(p,u)}
Primer je prikazan v \figref{bst-insert}. Najbolj časovno požrešen del tega procesa je začetno iskanje #x#-a, ki porabi količino časa, ki je sorazmerna z višino novo vstavljenega vozlišča #u#. V najslabšem primeru je ta enaka višini #BinarySearchTree#.


\begin{figure}
  \begin{center}
    \begin{tabular}{cc}
    \includegraphics[width=\HalfScaleIfNeeded]{figs/bst-example-4} &
    \includegraphics[width=\HalfScaleIfNeeded]{figs/bst-example-5} 
    \end{tabular}
  \end{center}
  \caption{Vstavljanje vrednosti $8.5$ v binarno iskalno drevo.}
  \figlabel{bst-insert}
\end{figure}


\subsection{Brisanje}

Brisanje vrednosti, ki jo hrani vozlišče #BinarySearchTree# #u#, je malce težje.  Če je #u# list, potem preprosto odstranimo #u# od njegovega starša.  Še bolje: če ima #u# samo enega otroka, potem lahko odstranimo #u# iz drevesa tako, da #u.parent# posvoji #u#-jevega otroka(glej
\figref{bst-splice}):
\codeimport{ods/BinarySearchTree.splice(u)}

\begin{figure}
  \begin{center}
    \includegraphics[scale=0.90909]{figs/bst-splice}
  \end{center}
  \caption{Brisanje lista ($6$) ali vozlišča z enim otrokom ($9$) je preprosto.}
  \figlabel{bst-splice}
\end{figure}

Reči se zakomplicirajo, ko pa ima #u# dva otroka.  V tem primeru je najlažje poiskati neko vozlišče #w#, ki ima manj kot dva otroka ter da #w.x# lahko zamenja #u.x#.  Za ohranjanje lastnosti binarnega iskalnega drevesa, mora biti vrednost #w.x# blizu vrednosti #u.x#.  Na primer, če bi izbrali #w# tako, da je #w.x# najmanjša vrednost, ki je večja od #u.x#, bi delovalo.  Iskanje primernega vozlišča #w# je preprosto; to je najmanjša vrednost, ki se nahaja v poddrevesu #u.right#.  To vozlišče lahko brez skrbi odstranimo, ker nima levega otroka (glej \figref{bst-remove}).
\codeimport{ods/BinarySearchTree.remove(u)}

\begin{figure}
  \begin{center}
    \begin{tabular}{cc}
    \includegraphics[width=\HalfScaleIfNeeded]{figs/bst-delete-1}
    \includegraphics[width=\HalfScaleIfNeeded]{figs/bst-delete-2}
    \end{tabular}
  \end{center}
  \caption[Brisanje iz Binarnega Iskalnega Drevesa]{Brisanje neke vrednosti ($11$) iz nekega vozlišča #u#, ki ima dva otroka, počnemo z zamenjavo #u#-jeve vrednosti z najmanjšo vrednostjo v #u#-jevem desnem poddrevesu.}
  \figlabel{bst-remove}
\end{figure}

\subsection{Povzetek}

Vsaka izmed funkcij #find(x)#, #add(x)# ter #remove(x)# v #BinarySearchTree# vključuje sledenje neki poti od korena drevesa pa do nekega vozlišča v drevesu. Brez dodatnega znanja o obliki drevesa je težko karkoli povedati o dolžini te poti, razen tega, da je pot manjša kot #n# - število vseh vozlišč v drevesu. Sledeči (nič kaj poseben) izrek povzame zmožnosti podatkovne strukture - #BinarySearchTree#:

\begin{thm}\thmlabel{bst}
  #BinarySearchTree# implementira #SSet# vmesnik ter podpira funkcije #add(x)#, #remove(x)# ter #find(x)# v $O(#n#)$ časa na operacijo.
\end{thm}

\thmref{bst} se slabo primerja z \thmref{skiplist}, ki prikazuje, da #SkiplistSSet# struktura lahko implementira #SSet# vmesnik z pričakovanim časom $O(\log #n#)$ na operacijo.  Problem #BinarySearchTree# tiči v tem, da lahko postane \emph{neuravnoteženo}. Namesto da drevo izgleda kot na \figref{bst}, lahko izgleda kot dolga veriga z #n# vozlišči, ki imajo po točno enega otroka, razen zadnjega,ki nima nobenega.

Obstaja več načinov kako se izogniti neuravnoteženim binarnim iskalnim drevesom. Vsi načini vodijo v podatkovne strukture, ki imajo operacije s časom $O(\log #n#)$. V \chapref{rbs} pokažemo kako lahko dosežemo operacije z \emph{pričakovanim} časom $O(\log #n#)$ s pomočjo naključnosti. V \chapref{scapegoat} pokažemo kako dosežemo operacije z \emph{amortiziranim} časom $O(\log #n#)$ s pomočjo delnih obnovitvenih operacij. V \chapref{redblack} pokažemo kako dosežemo operacije z \emph{najslabšim} časom $O(\log #n#)$ s pomočjo simulacije dreves, ki niso binarna: eno v katerem imajo vozlišča lahko do štiri otroke.
